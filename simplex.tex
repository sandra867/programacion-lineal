\documentclass{article}

\usepackage[spanish]{babel}
\usepackage{amsmath}
\usepackage[utf8]{inputenc}

\title{El Metodo Simplex}
\author{Sandra y Ana}
\begin{document}
\maketitle

 
\section{Introduccion}
\label{sec:introduccion}

El metodo simplex es un algoritmo para resolver porblemas de progrmacion
lineal.Fue inventado por George Dantzing en el año 1947.

\section{ejemplo}
\label{sec:ejemplo}

Ilustraremos la aplicacion del metodo simplex con un ejemplo.
 
 \begin{equation*} \begin{aligned} \text{Maximizar} \quad & 2x1+x2\\
     \text{sujeto a} \quad &
     \begin{aligned}
       x1-x2+x3 &= 2\\
       -2x1+x2+x4&=3\\
       3x1+4x2+x5&=12\\
       x1+x2+x6 &=1\\
       x1,x2,x3,x4,x5,x6 &\geq 0
     \end{aligned} \end{aligned} \end{equation*}

como en una de las desigualdades aparecen las variables del lado
izquierdo de un simbolo $\geq$, multiplicamos ambos miembros de esa
desigualdad por $-1$ para obtener la forma estandar
 \begin{equation*} \begin{aligned} \text{Maximizar} \quad & 2x1+x2\\
     \text{sujeto a} \quad &
     \begin{aligned}
       x1-x2+x3 &= 2\\
       -2x1+x2+x4&=2\\
       3x1+4x2+x5&=12\\
       x1+x2+x6 &=1\\
       x1,x2,x3,x4,x5,x6 &\geq 0
     \end{aligned} \end{aligned} \end{equation*}
para obtener la forma simplex, añadimos una variable de holgura por
cada desigualdad

 \begin{equation*} \begin{aligned} \text{Maximizar} \quad & 2x1+x2\\
     \text{sujeto a} \quad &
     \begin{aligned}
       x1-x2+x3 &= 2\\
       -2x1+x2+x4&=2\\
       3x1+4x2+x5&=12\\
       x1+x2+x6 &=1\\
       x1,x2,x3,x4,x5,x6 &\geq 0
     \end{aligned} \end{aligned} \end{equation*}

A continuacion obtenemos un tablero simplex despejando las variables
de holgura
 \begin{equation*} \begin{aligned} \text{Maximizar} \quad & 2x1+x2\\
     \text{sujeto a} \quad &
     \begin{aligned}
       x3 &= 2+x2-x1\\
       x4&=2+2x1-x2\\
       x5&=12-3x1-4x2\\
       x6 &=1-x1-x2\\
       \hline
       z&=\phantom{-1}-2x1
     \end{aligned} \end{equation*}


\end{document}